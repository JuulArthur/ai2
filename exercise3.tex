%ha med dette her
\documentclass[titlepage]{article}
\usepackage[english]{babel}
\usepackage[utf8]{inputenc}
\usepackage{parskip}
%\usepackage[latin1]{inputenc}
%\usepackage{graphicx}
%\usepackage{float}

%---- Listings --------------
\usepackage{color}
\definecolor{light-gray}{gray}{0.95}
\usepackage{listings}
\lstset{numbers=right,
      %numberstyle=\tiny,
      %numbers=left,
      %stepnumber=2,
      literate={æ}{{\ae}}
        {Æ}{{\AE}}1
        {ø}{{\o}}1
        {Ø}{{\O}}1
        {å}{{\aa}}1
        {Å}{{\AA}}1,
      firstnumber=1,
      numberfirstline=true
     breaklines=true,
     backgroundcolor=\color{light-gray},
     numbersep=5pt,
     xleftmargin=.25in,
     xrightmargin=.25in}
     \lstset{language=Java}

%Listings brukes slikt:
%\begin{lstlisting}{insert}
%Sett inn kode her
%\end{lstlisting}
%--------------------------

%hva som skal stå på tittelsida
\author{Juul Arthur Ribe Rudihagen and Tobias Linkjendal}
\title{Exercise 3}
\date{\today}

\begin{document}

%lager forsida
\maketitle

%\renewcommand{\abstractname}{Summary}
%man burde ha med et sammendrag
%\begin{abstract}
%Sammendrag av rapporten
%\end{abstract}

%man skal ha romertall på de innholdsfortegnelse
%før dette punktet skal man IKKE ha sidetall
\pagenumbering{roman}
\tableofcontents

%newpage gjør at du starter på neste side etterpå
\newpage

%herfra skal vi ha vanlige tall (arabiske (1,2,3,4,5...))
\pagenumbering{arabic}

%man pleier å starte med innledning
\section{Task 0}
We have chosen to model the decisions you can do when playing the computer game Smite. Smite is a MOBA (Multiplayer online battle arena) game where players cooperate and compete with each other over internett. Each game has 10 players, called gods, where 5 players are on each team. The goal of the game is to defeate the enemy gods and destroy their base. More specifically we have chosen the to model the decision of one of these players, when figthing against the other players. The question the player asks himself is "what should I do now?".

There are a lot of possibillities to do, and way to many variables to model. So we have limited this task to model only specefic parts of the game, and only take into consideration limited information. There are a lot of teamwork that can be done in this game, but we have choosen not to model this. We focus on the battle between us and one enemy god, and only model the other enemy as a group that can come and attack us.

\newpage


\section{Defining variables and decisions}

\subsection{Variables}
\subsubsection{Certain variables}
Here are the certain variables for our exercise. 


\subsubsection*{Health}
How much health do I have?

\subsubsection*{EnemyHealth}
How much health do my enemy have?

\subsubsection*{UltimateAttack}
Can I use my ultimate attack? This is a skill that is not always available and I might not have it active, but it is the skill I have that deals the most damage.

\subsubsection*{AlliedMinions}
Do I have minions with me that will attack my enemy? Minions are weak computer-controlled characters that attacks the enemy minions and gods.

\subsubsection*{EnemyMinions}
Do my enemy have minions that will attack me?



\subsubsection{Uncertain variables}
Here are the uncertain variables for our exercise. 

\subsubsection*{EnemyAttack}
Will the enemy attack? This is dependent on our health, the enemy's health and our minions. The stronger the enemy is and the weaker we are, the greater is the chance that he will attack.

\subsubsection*{EnemyUltimate}
Will the enemy use its ultimate attack? If I have little or medium health the enemy is more likely to use his ultimate in order to kill me. If the enemy has little health he is also more likely to use his ultimate in order to kill or stun me before I am able to kill him.

\subsubsection*{Ganked}
Will I be attacked my multiple enemies? This is dependent on my health and whether I have allied minions or not, because the weaker I am the easier I am to gank/be killed.

\subsubsection*{Survive}
It is not desirable to die. When you die you lose gold, give the enemy more experience and gold. You also leave your team more vulnarable while you are dead. If you have little health and are weaker than your enemy you have to be very careful not to die. This also means that you are more likely to retreat if you are close to death.

\subsubsection*{KillEnemy}
It is very desirable to kill the enemy. This will leave you stronger and the enemy will be weaker. There is a risk though going for a kill, because you can easily get killed yourself.

\subsubsection*{GetGold}
To get gold is crucial to win the game, the more gold you have the stronger your god becomes. You get gold by killing minions and attacking the enemy.


\subsection{Decisions}
The different actions to take are:

\subsubsection*{Attack}
This decision is wheter we want to attack our opponent or not. The stronger I am and the weaker the enemy is, the more likely I am to attack.
\subsubsection*{Retreat}
This decision is wheter we want to run away from the enemy or not. If I am weaker than my enemy and think I'm about to be attacked, I am more likely to retreat.
\subsubsection*{Attack minions}
Get gold by killing minions. If there are enemy minions, and I am weaker than my enemy or equally strong, I am more likely to just attack minions and get gold.

%\begin{itemize}
%\item Attack - This decision is wheter we want to attack our opponent or not.
%\item Retreat - This decision is wheter we want to run away from the enemy or not.
%\item Attack minions  - Farm up gold by killing minions
%\item Return to base - Go back to base to buy items and heal up
%\item Take jungle camps - Get more gold and exp by clearing jungle camps
%\item Go for gank - Help your teammates to kill enemy gods
%\end{itemize}



%slutt seksjon med newpage
\newpage

\section{The qualitative part}
We have modelled the exercise in GeNIe. Where we have the known variables in light blue, the unknown variables are orange, the utility function in dark blue and decisions in green. 

\begin{figure}[H]
\label{fig:Influencediagram}
\includegraphics[width=1px]{influencediagram.png}
\caption{Influence diagram}
\end{figure}

\newpage

\begin{thebibliography}{9}

\bibitem{telenorweb}
	Telenor,
	\emph{http://www.telenor.no}

\bibitem{netcomweb}
	Netcom,
	\emph{http://www.netcom.no}

\bibitem{foils}
	Dag Svanæs,
	\emph{Foiler medfølgende faget TDT4180}
\end{thebibliography}

\end{document}

