%ha med dette her
\documentclass[titlepage]{article}
\usepackage[english]{babel}
\usepackage[utf8]{inputenc}
\usepackage{parskip}
%\usepackage[latin1]{inputenc}
%\usepackage{graphicx}
%\usepackage{float}

%---- Listings --------------
\usepackage{color}
\definecolor{light-gray}{gray}{0.95}
\usepackage{listings}
\lstset{numbers=right,
      %numberstyle=\tiny,
      %numbers=left,
      %stepnumber=2,
      literate={æ}{{\ae}}
        {Æ}{{\AE}}1
        {ø}{{\o}}1
        {Ø}{{\O}}1
        {å}{{\aa}}1
        {Å}{{\AA}}1,
      firstnumber=1,
      numberfirstline=true
     breaklines=true,
     backgroundcolor=\color{light-gray},
     numbersep=5pt,
     xleftmargin=.25in,
     xrightmargin=.25in}
     \lstset{language=Java}

%Listings brukes slikt:
%\begin{lstlisting}{insert}
%Sett inn kode her
%\end{lstlisting}
%--------------------------

%hva som skal stå på tittelsida
\author{Juul Arthur Ribe Rudihagen and Tobias Linkjendal}
\title{Exercise 3}
\date{\today}

\begin{document}

%lager forsida
\maketitle

\renewcommand{\abstractname}{Summary}
%man burde ha med et sammendrag
\begin{abstract}
Sammendrag av rapporten
\end{abstract}

%man skal ha romertall på de innholdsfortegnelse
%før dette punktet skal man IKKE ha sidetall
\pagenumbering{roman}
\tableofcontents

%newpage gjør at du starter på neste side etterpå
\newpage

%herfra skal vi ha vanlige tall (arabiske (1,2,3,4,5...))
\pagenumbering{arabic}

%man pleier å starte med innledning
\section{Task 0}
We have chosen to model the decisions you can do when playing the computer game Smite. Smite is a MOBA game where players cooperate and compete with each other over internett. Each game has 10 players, called gods, where 5 players are on each team. The goal of the game is to defeate the enemy gods and destroy their base. More specifically we have chosen the to model the decision of one of these players, when figthing against the other players. The question the player asks himself is "what should I do now?".

There are a lot of possibillities to do, and way to many variables to model. So we have limited this task to model only specefic parts of the game, and only take into consideration limited information. There are a lot of teamwork that can be done in this game, but we have choosen to not model this.

\newpage


\section{Step1}
KanskewrwerwLOL

\subsection{Step 1a}
\subsubsection{Certain variables}
\subsubsection*{Level}
Do I have a higher level than my enemy?

\subsubsection*{Health}
Do I have more health than my enemy?

\subsubsection*{AttackPower}
Do I have more attack power than my enemy?

\subsubsection*{DefencePower}
Do I have more defence power than my enemy

\subsubsection*{Mana}
Do I have enough mana for ultimate attack? 

\subsubsection*{Minions}
Do I have minions with me?

\subsubsection*{EnemyMinions}
Do my enemy have minions?




\subsubsection{Uncertain variables}
\subsubsection*{AttackHit}
TF

\subsubsection*{CriticalHit}
T

\subsubsection*{EnemyAction}
T


\subsubsection*{}
T

\subsubsection*{}
T

\subsection{Steb 1b}
The diffrent actions to take are:
\subsubsection*{Attack}
This decision is wheter we want to attack our opponent or not.
\subsubsection*{Retreat}
This decision is wheter we want to run away from the enemy or not.
\subsubsection*{Attack minions}
Get gold by killing minions.
\subsubsection*{Return to base}
Go back to base to buy items and heal up.
\subsubsection*{Take jungle camps}
Get more gold and exp by clearing jungle camps
\subsubsection*{Go for gank}
Help your teammates to kill enemy gods
%\begin{itemize}
%\item Attack - This decision is wheter we want to attack our opponent or not.
%\item Retreat - This decision is wheter we want to run away from the enemy or not.
%\item Attack minions  - Farm up gold by killing minions
%\item Return to base - Go back to base to buy items and heal up
%\item Take jungle camps - Get more gold and exp by clearing jungle camps
%\item Go for gank - Help your teammates to kill enemy gods
%\end{itemize}

\subsubsection{Underunderseksjon}
Bla bla

%slutt seksjon med newpage
\newpage

\section*{Seksjon som ikke er nummerert}
Blabla
%\\ %tvinger frem en ny linje, kan kombineres til flere \\\\ => to nye linjer
``Her kommre ting i hermetegn'' %forferdelig at det er disse to tegna som brukes, men sånn er det
%\\\\
%\\\\
\textbf{Tykk tekst}

%%lag figur

%\begin{figure}[H]
%\label{fig:logo}
%\includegraphics[width=1px]{navn.png}
%\caption{chriserdigg}
%\end{figure}
%\\

%%inkluder figur
%\includegraphics[width=0.25,angle=45]{navn.png}

%tabell
\begin{table}[H]
   \centering
   \label{tab:lulz}
   \begin{tabular}{| l | c | r |}
      \hline
      1 & 2 & 3 \\
      \hline
      4 & 5 & 6 \\
      \hline
      7 & 8 & 9 \\
      \hline
   \end{tabular}
   \caption{verdens beste tabell}
\end{table}


%inkludért seksjon, hva vi burde bruke
%\input{filnavn.tex}
%input setter inn teksten som er i filnavn.tex, og vi kan dermed style i dette dokumentet

\newpage
%referanseliste
\begin{thebibliography}{9}

\bibitem{telenorweb}
	Telenor,
	\emph{http://www.telenor.no}

\bibitem{netcomweb}
	Netcom,
	\emph{http://www.netcom.no}

\bibitem{foils}
	Dag Svanæs,
	\emph{Foiler medfølgende faget TDT4180}
\end{thebibliography}

\end{document}

